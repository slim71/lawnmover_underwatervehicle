\documentclass{article}

\usepackage[utf8]{inputenc}
\usepackage[italian]{babel}
\usepackage[T1]{fontenc}
\usepackage{amsmath} 
\usepackage{latexsym}
\usepackage{graphicx}
\usepackage{gensymb}
\usepackage{textcomp}
\usepackage{imakeidx}
\usepackage{siunitx}
\usepackage{amssymb}
\usepackage{biblatex}
\usepackage{csquotes}
\usepackage{hyperref}

\addbibresource{../references.bib}
\graphicspath{{../Immagini/} {./}}

\title{Report Subacquei, Gruppo A1\\
Esplorazione fondale}
\author{Canonico Martina, Scumaci Cristina, Vollaro Simone}
\date{June 2020}

%TODO: accenti

\begin{document}\pagenumbering{arabic}
    \input{"Frontespizio subacquei"}
    
    \newpage
    \tableofcontents
    \thispagestyle{empty}
    
    
    \newpage
    
    \section{Introduzione1}
    La nostra missione consiste nell'esplorazione di un'area con fondale irregolare: si tratta di una survey di tipo lawn-mower e il task
    di esecuzione prevede che il veicolo debba mantenere una velocità di crociera costante, effettuando l'esplorazione con altitudine 
    (distanza del veicolo rispetto al fondale) mantenuta ad un valore desiderato.
    \\
    \noindent
    Il compito del nostro modulo è stato quello di definire istante per istante lo stato della missione (e relative operazioni elementari) attualmente attivo;
    l’aggiornamento di tale stato è stato effettuato sulla base del monitoraggio di alcune variabili, tra cui un array di flag
    chiamato \textit{"current\_state"}, di visibilità globale così che tutti i moduli avessero l'informazione dello stato attuale e
    potessero conformemente adeguare il proprio comportamento.
	
    \vspace{\baselineskip}
    In quanto modulo di references generator, il nostro compito è stato anche quello di generare i segnali di riferimento relativi a posizione,
    orientazione e velocità delle varie operazioni elementari all'interno dei vari task, per il corretto svolgimento della missione.
 q       
    \newpage
    \section{sec opage}!
    aksdfnbsfddosnjd
    \hyperref[hihi]{PRova link}\cite{stateflow}
    issues
    \newpage
    \section{3rd}%!ciao
    hi\label{hihi}
    
    \newpage
    \printbibliography
\end{document}

